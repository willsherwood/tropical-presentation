\section{The Section}

\begin{frame}
  \frametitle{Mechanism Design}

        \begin{enumerate}
  \item Let $X$ be a set of types, and $O$ a set of outcomes. 
        
        \item Let $T : X \times O \to \mathbb R$ be 
  a type utility map, specifying utility: 
                        \begin{enumerate} \item If the player's true type is $t^*$, and outcome 
                        $o$ is implemented, the player will receive $T(t^*, o)$ utility.
                        \end{enumerate}
        \end{enumerate}

  A mechanism designer will publish maps $g : X \to O$, $p : O \to \mathbb R$
        called the mechanism.

  $g$ specifies which outcome will happen for each possible declared type. 
  $p$ specifies for each possible outcome, how much utility will be charged to the player.
        
  For a player of true type $t^*$, the player will look at the $g$ and $p$, declare a type $x$, 
        and receieve utility $$ T(t^*, g(x)) - p(g(x)).$$

        Our goal is to classify the mechanisms $(g,p)$ for which the player should always
        declare their true type $t^*$.
\end{frame}

\begin{frame}
        \frametitle{Incentive Compatible}
        We take a slightly weaker stance in defining incentive compatible.
        A player could declare a type other than $t^*$, but
        if they do, they must get the same overall utility as if they declared $t^*$.

        Assuming utility-rationality, the player will be indifferent to declaring their true type 
        $t^*$ exactly when for all $x \in X$,

        $$T(t^*, g(x)) - p(g(x)) \le T(t^*, g(t^*)) - p(g(t^*)).$$
\end{frame}

\begin{frame}

\end{frame}

\begin{frame}
  \frametitle{Functional derivative}
  A functional derivative  $\frac{\partial F}{\partial\mathbf{x}}$ is implicitely defined by the equation
  \begin{align*}
    dF\left[\mathbf{x},\tilde{\mathbf{x}}\right]=\int_{\Omega}d\Omega\,\tilde{\mathbf{x}}\frac{\partial F}{\partial\mathbf{x}},
  \end{align*}
  where the functional differential $dF\left[\mathbf{x},\tilde{\mathbf{x}}\right]$ is defined as
  \begin{align*}
    dF\left[\mathbf{x},\tilde{\mathbf{x}}\right]\equiv\left.\frac{d}{d\varepsilon}\right|_{\varepsilon=0}F\left[\mathbf{x}+\varepsilon\tilde{\mathbf{x}}\right].
  \end{align*}
\end{frame}
